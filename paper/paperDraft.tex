\documentclass[11pt]{article}

\usepackage{graphicx}

\begin{document}

\section{Introduction}
Applications for computer vision and methods for processing images have considerably benefited in recent years from developments made possible by deep learning and artificial intelligence (AI). One of these is picture synthesis, which is the act of creating new images and modifying ones that already exist. Because to its many useful applications in fields including art creation, image editing, virtual reality, video games, and computer-aided design, image synthesis is a fascinating and significant field of research.\\
Text-conditional image models are capable to generate images from text queries and can arrange unrelated objects in a semantically plausible way. They are also called text-to-image models.
One of the most popular examples of such models is Open AI's Dall-E 2.

\begin{figure} [h]
	\includegraphics[width = 1\linewidth]{images/DallEExample.png}
	\caption{On the left 4 generated pictures to the text input: "a tapir with the texture of an accordion" and on the right 4 generated pictures to the text input: "an illustration of a baby hedgehog in a christmas walking a dog" \cite{zeroShot}}
	\label{example}
\end{figure}

\subsection{Ethical approaches}

\subsubsection{Kant}
The german philosopher Immanuel Kant (1724--1804) introduced the categorical imperative in his book "Groundwork of the Metaphysic of Morals" in 1785. \\
Kant formulated the categorical imperative as followed: "Act only according to that maxim by which you can at the same time will that it should become a universal law", implying that you should only behave a specific way if you want everyone else to do the same.

\subsubsection{Utilitarianism}
Utilitarianism is an ethical theory that delineates right from wrong by focusing on outcomes. It is a form of consequentialism, the right action is understood entirely in terms of consequences produced.
Utilitarianism assumes that the most ethical decision is the one that produces the greatest good for the greatest number of people. It is based mostly on ideas from Jeremy Bentham (1748--1832) and John Stuart Mill (1806--1873). They  equated the good and pleasure.\\
It is the only moral system that can be used to defend using force or going to war. Due to the way it takes advantages and costs into account, it is also the method of moral reasoning that is most frequently applied in business. \cite{EthicsUnwrapped}. \\
But utilitarian ethical decision-making has its limits. Often it is not explicitly certain in advance what the exact consequences of an action will be. 
An often cited example is that there are four people in a hospital who all need an organ donation of a different organ. According to utilitarianism, the right decision would be to sacrifice one healthy person and donate his organs to save the other four.

\section{Technical Background}

\section{Ethical Consideration}
\subsection{Kant}
\subsection{Utilitarianism}
According to Utilitarianism the good or pleasure originated of the AI-generated images has to be taken into account to decide if its an ethically justifiable action. \\
The pleasure derived from the generated picture of the person that entered the query is the only one that has to be taken into account if its not made public. 
There are different scenarios that can occur where from a utilitarian point of view it
\section{What is done?}
\section{What could be done?}
\section{Conclusion}

\bibliographystyle{plain}
\bibliography{bib.bib}
\end{document}
